% Uncomment for bibliography using biblatex (see includes/formatting.tex)
% \section*{Bibliography}
% Print every citation in citations.bib, even if unused by \autocite
% \nocite{*}
% \printbibliography[heading=none]

\section*{Research credit}
Special thanks to Giuseppe Stelluto for his demonstration of an alternative method for deriving the formula for the sequences $L_n$ and $W_n$. His approach was much more technically rigorous, but I felt that my approach was more accessible to those without a formal education in mathematics.

Additional thanks to \href{https://www.papersizes.org}{papersizes.org} for supplying an CSV (comma seperated values) file with the scaling factors. Originally I was making the table by hand in LaTeX, but that literally took forever. With an premade CSV file, I was able to work much faster, and simply convert the CSV format to a \LaTeX table. 

\section*{Image credit}
\href{https://en.wikipedia.org/wiki/File:A_size_illustration2.svg}{Cover page illustration is a diagram illustrating ISO 216 A-series paper sizes, sourced from Wikipedia under Creative Commons (CC BY-SA 3.0) license.}

\section*{Technical Notes}
This essay is typeset using \LaTeX, an Open Source document typesetting language
by Donald Knuth, and version-controlled via Git. The git repository containing notes, source code, and revision history is available upon request.

% Optional: Include github URL here

\noindent
This essay is written using the EssayTemplate, an open source \LaTeX\ essay
template designed for the Humanities by Shen Zhou Hong. It is available at:

https://github.com/ShenZhouHong/EssayTemplate

\vfill
\begin{center}
This \LaTeX\ essay is also available in Microsoft Word, OpenOffice, HTML, and \mbox{plain text} upon request.
\end{center}
